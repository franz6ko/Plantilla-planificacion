\documentclass[11pt]{charter}

% El títulos de la memoria, se usa en la carátula y se puede usar el cualquier lugar del documento con el comando \ttitle
\titulo{Manejo Autónomo de Maquinaria Agricola Utilizando Visión e Inteligencia Artificial - Multiview CNNs} 

% Nombre del posgrado, se usa en la carátula y se puede usar el cualquier lugar del documento con el comando \degreename
%\posgrado{Carrera de Especialización en Sistemas Embebidos} 
%\posgrado{Carrera de Especialización en Internet de las Cosas} 
\posgrado{Carrera de Especialización en Intelegencia Artificial}
%\posgrado{Maestría en Sistemas Embebidos} 
%\posgrado{Maestría en Internet de las cosas}

% Tu nombre, se puede usar el cualquier lugar del documento con el comando \authorname
\autor{Francisco Tinelli} 

% El nombre del director y co-director, se puede usar el cualquier lugar del documento con el comando \supname y \cosupname y \pertesupname y \pertecosupname
\director{Pablo Farreras}
\pertenenciaDirector{CEO de SUPORTRONICS SRL} 
% FIXME:NO IMPLEMENTADO EL CODIRECTOR ni su pertenencia
\codirector{} % si queda vacio no se deberíá incluir 
\pertenenciaCoDirector{}

% Nombre del cliente, quien va a aprobar los resultados del proyecto, se puede usar con el comando \clientename y \empclientename
\cliente{Pablo Farreras}
\empresaCliente{SUPPORTRONICS}

% Nombre y pertenencia de los jurados, se pueden usar el cualquier lugar del documento con el comando \jurunoname, \jurdosname y \jurtresname y \perteunoname, \pertedosname y \pertetresname.
\juradoUno{Nombre y Apellido (1)}
\pertenenciaJurUno{pertenencia (1)} 
\juradoDos{Nombre y Apellido (2)}
\pertenenciaJurDos{pertenencia (2)}
\juradoTres{Nombre y Apellido (3)}
\pertenenciaJurTres{pertenencia (3)}
 
\fechaINICIO{05 de marzo de 2021}		%Fecha de inicio de la cursada de GdP \fechaInicioName
\fechaFINALPlanificacion{30 de abril de 2021} 	%Fecha de final de cursada de GdP
\fechaFINALTrabajo{?? de diciembre de 2021}		%Fecha de defensa pública del trabajo final


\begin{document}

\maketitle
\thispagestyle{empty}
\pagebreak


\thispagestyle{empty}
{\setlength{\parskip}{0pt}
\tableofcontents{}
}
\pagebreak


\section{Registros de cambios}
\label{sec:registro}


\begin{table}[ht]
\label{tab:registro}
\centering
\begin{tabularx}{\linewidth}{@{}|c|X|c|@{}}
\hline
\rowcolor[HTML]{C0C0C0} 
Revisión & \multicolumn{1}{c|}{\cellcolor[HTML]{C0C0C0}Detalles de los cambios realizados} & Fecha      \\ \hline
1.0      & Creación del documento                                          & 27/06/2020 \\ \hline
1.1      & Avances en alguna cosa                                          & dd/mm/aaaa \\ \hline
1.2      & Otro ejemplo \newline
		   Con texto partido \newline
		   En varias líneas \newline
		   A propósito                                                     & dd/mm/aaaa \\ \hline
\end{tabularx}
\end{table}

\pagebreak



\section{Acta de constitución del proyecto}
\label{sec:acta}

\begin{flushright}
Buenos Aires, \fechaInicioName
\end{flushright}

\vspace{2cm}

Por medio de la presente se acuerda con el Ing. \authorname\hspace{1px} que su Trabajo Final de la \degreename\hspace{1px} se titulará ``\ttitle'', consistirá esencialmente en el prototipo preliminar de un sistema de manejo autonomo de maquinaria agricola mediante la utilizacion de vision e inteligencia artificial, y tendrá un presupuesto preliminar estimado de 600 hs de trabajo y recursos economicos bridnados por la empresa, con fecha de inicio \fechaInicioName\hspace{1px} y fecha de presentación pública \fechaFinalName.

Se adjunta a esta acta la planificación inicial.

\vfill

% Esta parte se construye sola con la información que hayan cargado en el preámbulo del documento y no debe modificarla
\begin{table}[ht]
\centering
\begin{tabular}{ccc}
\begin{tabular}[c]{@{}c@{}}Ariel Lutenberg \\ Director posgrado FIUBA\end{tabular} & \hspace{2cm} & \begin{tabular}[c]{@{}c@{}}\clientename \\ \empclientename \end{tabular} \vspace{2.5cm} \\ 
\multicolumn{3}{c}{\begin{tabular}[c]{@{}c@{}} \supname \\ Director del Trabajo Final\end{tabular}} \vspace{2.5cm} \\
%\begin{tabular}[c]{@{}c@{}}\jurunoname \\ Jurado del Trabajo Final\end{tabular}     &  & \begin{tabular}[c]{@{}c@{}}\jurdosname\\ Jurado del Trabajo Final\end{tabular}  \vspace{2.5cm}  \\
%\multicolumn{3}{c}{\begin{tabular}[c]{@{}c@{}} \jurtresname\\ Jurado del Trabajo Final\end{tabular}} \vspace{.5cm}                                                                     
\end{tabular}
\end{table}




\section{Descripción técnica-conceptual del proyecto a realizar}
\label{sec:descripcion}

\begin{itemize}
\item Misión de la Empresa
\end{itemize}

Supportronics SRL se enfoca en aumentar la competitividad de las empresas de sus clientes, a través de innovación tecnológica en electrónica y software.

La experiencia técnica y focalización en dirección de proyectos de su equipo, permite obtener la mejor relación costo-calidad. Además, posee socios estratégicos localizadas en Estados Unidos, Alemania y China, posibilitando la optimización de todos los procesos de interacción y comunicación.

\begin{itemize}
\item Vinculación del proyecto
\end{itemize}

Dicho proyecto está directamente vinculado con la empresa ya que integra muchos de los conceptos antes mencionados obteniendo como resultado un producto final competente para el mercado.

Este proyecto consistirá en la mejora de un producto comercial del cliente que ya se encuentra disponible en el mercado hace varios años. El mismo ya ha pasado por varias etapas de evolución originados del feedback obtenido de sus clientes. En esta etapa, se estudiará la factibilidad de agregar la funcionalidad de manejo autónomo.

Actualmente el producto ofrecido por la empresa se encuentra entre las etapas de madurez y declinación del ciclo de vida del producto. Esto es así debido a que el funcionamiento de la máquina es estrictamente mecánico, sin ninguna intervención electrónica o informática, lo que la pone en desventaja frente a sus competidores.

El objetivo de este proyecto es modernizar el producto del cliente para incorporar tecnología y reubicar el mismo en un nuevo ciclo de vida del producto mejorado.

\begin{itemize}
\item Origen del proyecto - Necesidad
\end{itemize}

Debido a la naturaleza demandante del trabajo humano en el sector agrícola, cada vez es menor la cantidad de gente que quiere dedicarse al mismo. Las intensas tareas implicadas conllevan condiciones físicas desgastantes que limitan el trabajo a la gente más apta (generalmente hombres jóvenes o de mediana edad) y conllevan a un deterioro humano prematuro. Por ello, cada vez existe una mayor necesidad de tecnificar y realizar el trabajo
de forma automática y eficiente. 

Es objetivo del presente proyecto incorporar tecnología para mejorar esas condiciones de trabajo y extender la posibilidad de trabajo a personas que de otra manera no podrían, ya sea por razones de edad, discapacidad o falta de condiciones físicas entre otras.

Para ello, Supportronics se propone desarrollar e implementar un sistema de manejo autónomo en agricultura extensiva para implementar en maquinaria agrícola autopropulsada permitiendo el desplazamiento de la misma entre las hileras del cultivo sin la necesidad de la intervención humana.

Actualmente existen diversas tecnologías de manejo autónomo para lo que se denomina agricultura intensiva como son la soja, maíz, girasol,etc. Por otra parte existe otra rama de la agricultura que vendría siendo la extensiva como puede ser lechuga, durazno, viñas, etc. En este tipo de agricultura no se encuentran maquinaria comercial apta para realizar un manejo autónomo, siendo este sector muy vulnerable ante las exigencias del personal de trabajo. 

En la Figura \ref{fig:diagBloques} se presenta el diagrama de bloques del sistema. Se observa en naranja la etapa de creación y entrenamiento de la red neuronal donde la cámara de visión artificial genera un dataset de imágenes en el cual deben ser claramente visualizados los distintos entornos del mismo, las distintas plantaciones y su disposición en el espacio. El bloque de entrenamiento es alimentado de este dataset y del diseño de red neuronal propuesto, luego se procede con el entrenamiento de la misma hasta obtener un resultado final. Si el error relativo no es aceptable se procede a generar un nuevo diseño de red y modificar los datos datos generados, en caso de ser un error aceptable se procede a conformar la red. En el bloque azul ya se procede a su test e implementación en campo, donde nuevamente el entorno es adquirido mediante la cámara de visión para ser enviada al procesador de inteligencia artificial que ya posee una red neuronal entrenada y validada. El resultado obtenido es aportado al sistema de navegación para la toma de decisiones del control de dirección y el control de avance.

\vspace{60px}

\begin{figure}[htpb]
\centering 
\includegraphics[width=.99\textwidth]{./Figuras/Diagrama1.png}
\caption{Diagrama en bloques del sistema}
\label{fig:diagBloques}
\end{figure}

\vspace{60px}

\section{Identificación y análisis de los interesados}
\label{sec:interesados}

\begin{consigna}{black} 

\begin{table}[ht]
%\caption{Identificación de los interesados}
%\label{tab:interesados}
\begin{tabularx}{\linewidth}{@{}|l|X|X|l|@{}}
\hline
\rowcolor[HTML]{C0C0C0} 
Rol           & Nombre y Apellido & Organización 	& Puesto 	\\ \hline
%Auspiciante  &                   &					&        	\\ \hline
Cliente       & \clientename      &\empclientename	& CEO       	\\ \hline
%Impulsor     &                   &              	&        	\\ \hline
Responsable   & \authorname       & Supportronics  	& Alumno 	\\ \hline
%Colaboradores & -                 &    				&       	\\ \hline
Orientador    & \supname	      & \pertesupname 	& Director	Trabajo final \\ \hline
Equipo        & Francisco Tinelli
	   \newline Federico Sesto 
  	   \newline Santiago Rivier          & SUPPORTRONICS & Desarrolladores 	\\ \hline
%Opositores    &                   &              	&        	\\ \hline
%Usuario final &                   &              	&        	\\ \hline
\end{tabularx}
\end{table}

\begin{itemize}
\item Cliente: Pablo Farreras, es quien nos porporciona el proyecto.
\item Responsable: Francisco Tinelli, es el que va a dasarrollar el proyecto junto con su equipo de trabajo.
\item Orientador: Pablo Farreras, es el jefe del proyecto y lider del equipo.
\item Equipo: Federico Sesto y Santiago Rivier, son el equipo con quien se va a trabajar para llevar a cabo el proyecto.
\end{itemize}

\end{consigna}



\section{1. Propósito del proyecto}
\label{sec:proposito}

Desarrollar e incorporar un sistema de manejo autónomo para maquinaria agrícola mediante la utilización de una cámara de profundidad e inteligencia artificial, para mejorar la competitividad del producto del cliente y lograr una mayor inclusividad de personal de campo mejorando sustancialmente sus condiciones de trabajo.


\section{2. Alcance del proyecto}
\label{sec:alcance}

Debido a la gran magnitud del proyecto propuesto por la empresa, el presente proyecto final destinado a la especialización comprenderá solo una primera etapa del gran proyecto, donde es de mayor importancia la aplicación de inteligencia artificial.

El proyecto comprenderá los siguientes puntos y etapas:

\begin{itemize}
\item Investigación sobre las tecnologías a utilizar (Cámara y Neural Stick).
\item Formación de know-how sobre las mismas a través del estudio y aplicación en pequeños ejemplos de prueba / demos.
\item Formación en las áreas de inteligencia artificial que se aplicarán en el proyecto.
\item Investigación y aprendizaje sobre las diferentes estrategias de manejo autónomo existentes y utilizadas en la actualidad.
\item Extraccion y etiquetado de datos para la conformacion de un dataset.
\item Selección y diseño de la arquitectura del sistema de manejo autónomo aplicada al caso particular del cliente del proyecto.
\item Desarrollo del sistema de detección y procesamiento del entorno para orientar la maquinaria en el recorrido entre las hileras de cultivo.
\end{itemize}

Puntos no comprendidos en el proyecto:

\begin{itemize}
\item Desarrollo del sistema de navegación.
\item Desarrollo de los sistemas de control de los actuadores de la maquinaria.
\item Aplicación en el campo de la solución elaborada.
\end{itemize}

\section{3. Supuestos del proyecto}
\label{sec:supuestos}

Para el desarrollo del presente proyecto se supone que primero es necesario realizar un know-how de las tecnologías necesarias a implementar en el producto.

Ademas se supone que es necesario generar una base de conocimientos en Inteligencia Artificial, enfocado en el area de Visión artificial y Manejo Autónomo.

Otra suposición importante a tener en cuenta es la disponibilidad de tiempo ya que dicho proyecto cuenta con una fecha de entrega establecida.

Se supone que se cuenta con el hardware necesario para llevar a cabo todo el desarrollo.


\section{4. Requerimientos}
\label{sec:requerimientos}

\begin{enumerate}
\item Grupo de requerimientos asociados con la vision artificial
	\begin{enumerate}
	\item Analizar capacidades de la camara
	\item Implementar codigos de ejemplo
	\item Generar dataset de imágenes
	\item Desarrollar método de visión artificial
	\item Implementar método desarrollado.
	\item Validar método de visión artificial
	\end{enumerate}
\item Grupo de requerimientos asociados con Neural Stick
	\begin{enumerate}
	\item Analizar capacidades
	\item Implementar programas de ejemplo
	\item Analizar posibilidad de integrar en display
	\item Analizar viabilidad para el uso continuo.
	\end{enumerate}
\end{enumerate}


\section{Historias de usuarios (\textit{Product backlog})}
\label{sec:backlog}

\begin{consigna}{red}
Descripción: En esta sección se deben incluir las historias de usuarios y su ponderación (\textit{history points}). Recordar que las historias de usuarios son descripciones cortas y simples de una característica contada desde la perspectiva de la persona que desea la nueva capacidad, generalmente un usuario o cliente del sistema. La ponderación es un número entero que representa el tamaño de la historia comparada con otras historias de similar tipo.
\end{consigna}

\section{5. Entregables principales del proyecto}
\label{sec:entregables}

Cómo resultado del presente proyecto, se entregará a los docentes correspondientes de la Facultad de Ingeniería de la UBA, el informe final del proyecto con firma previa de los documentos de confidencialidad.

En cuanto al cliente, también se le entregará el informe final junto con el código fuente además de diferentes programas y documentos adicionales que reflejarán el progreso del proyecto a medida que el mismo avance.

\section{6. Desglose del trabajo en tareas}
\label{sec:wbs}

\begin{enumerate}
\item Vision computacional
	\begin{enumerate}
	\item Búsqueda de información y documentación (40 hs)
	\item Integración con placa y display (40 hs)
	\item Reconocimiento de objetos con código ejemplo (40 hs)
	\item Extraccion de caracteristicas para generar dataset. (40 hs)
	\item Desarrollo de algoritmo de visión artificial (40 hs)
	\item Pruebas de funcionamiento en placa y display (40 hs)
	\end{enumerate}
\item Unidad de procesamiento Neural
	\begin{enumerate}
	\item Búsqueda de información y documentación (40 hs)
	\item Implementación de códigos ejemplos (40 hs)
	\item Integración con placa y display (40 hs)
	\item Reconocimiento de objetos implementado en la palca y display (40 hs).
	\end{enumerate}
\item Generación de red neuronal
	\begin{enumerate}
	\item Investigar tipos de redes neuronales (40 hs)
	\item Analizar ventajas y desventajas para la tarea a ralizar (40 hs)
	\item Desarrollar computacionalmente la red neuronal seleccionada (40 hs)
	\item Analizar la capacidad de implementar dicha red neuronal en placa y display (40 hs)
	\item Entrenamiento de la red neuronal (40 hs)
	\item Realizar integración de red neuronal con display, cámara y unidad de procesamiento Neural (40 hs)
	\end{enumerate}
\end{enumerate}

Cantidad total de horas: (640 hs)


\section{7. Diagrama de Activity On Node}
\label{sec:AoN}

\begin{consigna}{red}
Armar el AoN a partir del WBS definido en la etapa anterior. 

%La figura \ref{fig:AoN} fue elaborada con el paquete latex tikz y pueden consultar la siguiente referencia \textit{online}:

%\url{https://www.overleaf.com/learn/latex/LaTeX_Graphics_using_TikZ:_A_Tutorial_for_Beginners_(Part_3)\%E2\%80\%94Creating_Flowcharts}

\end{consigna}

\begin{figure}[htpb]
\centering 
\includegraphics[width=.8\textwidth]{./Figuras/AoN.png}
\caption{Diagrama en \textit{Activity on Node}}
\label{fig:AoN}
\end{figure}

Indicar claramente en qué unidades están expresados los tiempos.
De ser necesario indicar los caminos semicríticos y analizar sus tiempos mediante un cuadro.
Es recomendable usar colores y un cuadro indicativo describiendo qué representa cada color, como se muestra en el siguiente ejemplo:



\section{8. Diagrama de Gantt}
\label{sec:gantt}

\begin{consigna}{red}
Utilizar el software Gantter for Google Drive o alguno similar para dibujar el diagrama de Gantt.

Existen muchos programas y recursos \textit{online} para hacer diagramas de gantt, entre las cuales destacamos:

\begin{itemize}
\item Planner
\item GanttProject
\item Trello + \textit{plugins}. En el siguiente link hay un tutorial oficial: \\ \url{https://blog.trello.com/es/diagrama-de-gantt-de-un-proyecto}
\item Creately, herramienta online colaborativa. \\\url{https://creately.com/diagram/example/ieb3p3ml/LaTeX}
\item Se puede hacer en latex con el paquete \textit{pgfgantt}\\ \url{http://ctan.dcc.uchile.cl/graphics/pgf/contrib/pgfgantt/pgfgantt.pdf}
\end{itemize}

Pegar acá una captura de pantalla del diagrama de Gantt, cuidando que la letra sea suficientemente grande como para ser legible. 
Si el diagrama queda demasiado ancho, se puede pegar primero la ``tabla'' del Gantt y luego pegar la parte del diagrama de barras del diagrama de Gantt.

Configurar el software para que en la parte de la tabla muestre los códigos del EDT (WBS).\\
Configurar el software para que al lado de cada barra muestre el nombre de cada tarea.\\
Revisar que la fecha de finalización coincida con lo indicado en el Acta Constitutiva.

En la figura \ref{fig:gantt}, se muestra un ejemplo de diagrama de gantt realizado con el paquete de \textit{pgfgantt}. En la plantilla pueden ver el código que lo genera y usarlo de base para construir el propio.

\begin{figure}[htbp]
\begin{center}
\begin{ganttchart}{1}{12}
  \gantttitle{2020}{12} \\
  \gantttitlelist{1,...,12}{1} \\
  \ganttgroup{Group 1}{1}{7} \\
  \ganttbar{Task 1}{1}{2} \\
  \ganttlinkedbar{Task 2}{3}{7} \ganttnewline
  \ganttmilestone{Milestone o hito}{7} \ganttnewline
  \ganttbar{Final Task}{8}{12}
  \ganttlink{elem2}{elem3}
  \ganttlink{elem3}{elem4}
\end{ganttchart}
\end{center}
\caption{Diagrama de gantt de ejemplo}
\label{fig:gantt}
\end{figure}

\end{consigna}

\section{9. Matriz de uso de recursos de materiales}
\label{sec:recursos}


\begin{table}
\label{tab:recursos}
\centering
\begin{tabularx}{\linewidth}{@{}|c|X|X|X|X|c|@{}}
\hline
\cellcolor[HTML]{C0C0C0} & \cellcolor[HTML]{C0C0C0} & \multicolumn{4}{c|}{\cellcolor[HTML]{C0C0C0}Recursos requeridos (horas)} \\ \cline{3-6} 
\multirow{-2}{*}{\cellcolor[HTML]{C0C0C0}\begin{tabular}[c]{@{}c@{}}Código\\ WBS\end{tabular}} & \multirow{-2}{*}{\cellcolor[HTML]{C0C0C0}\begin{tabular}[c]{@{}c@{}}Nombre \\ tarea\end{tabular}} & Material 1 & Material 2 & Material 3 & Material 4 \\ \hline
 &  &  &  &  &  \\ \hline
 &  &  &  &  &  \\ \hline
 &  &  &  &  &  \\ \hline
 &  &  &  &  &  \\ \hline
 &  &  &  &  &  \\ \hline
 &  &  &  &  &  \\ \hline
 &  &  &  &  &  \\ \hline
 &  &  &  &  &  \\ \hline 
 &  &  &  &  &  \\ \hline
 &  &  &  &  &  \\ \hline
 &  &  &  &  &  \\ \hline
 &  &  &  &  &  \\ \hline
 &  &  &  &  &  \\ \hline
 &  &  &  &  &  \\ \hline
 &  &  &  &  &  \\ \hline
 &  &  &  &  &  \\ \hline
 &  &  &  &  &  \\ \hline
 &  &  &  &  &  \\ \hline
 &  &  &  &  &  \\ \hline
 &  &  &  &  &  \\ \hline
 &  &  &  &  &  \\ \hline
 &  &  &  &  &  \\ \hline
 &  &  &  &  &  \\ \hline
 &  &  &  &  &  \\ \hline 
 &  &  &  &  &  \\ \hline
 &  &  &  &  &  \\ \hline
 &  &  &  &  &  \\ \hline
 &  &  &  &  &  \\ \hline

\end{tabularx}%
\end{table}


\section{10. Presupuesto detallado del proyecto}
\label{sec:presupuesto}

\begin{consigna}{red}
Si el proyecto es complejo entonces separarlo en partes:
\begin{itemize}
\item Un total global, indicando el subtotal acumulado por cada una de las áreas.
\item El desglose detallado del subtotal de cada una de las áreas.
\end{itemize}

IMPORTANTE: No olvidarse de considerar los COSTOS INDIRECTOS.

\end{consigna}

\begin{table}[htpb]
\centering
\begin{tabularx}{\linewidth}{@{}|X|c|r|r|@{}}
\hline
\rowcolor[HTML]{C0C0C0} 
\multicolumn{4}{|c|}{\cellcolor[HTML]{C0C0C0}COSTOS DIRECTOS} \\ \hline
\rowcolor[HTML]{C0C0C0} 
Descripción &
  \multicolumn{1}{c|}{\cellcolor[HTML]{C0C0C0}Cantidad} &
  \multicolumn{1}{c|}{\cellcolor[HTML]{C0C0C0}Valor unitario} &
  \multicolumn{1}{c|}{\cellcolor[HTML]{C0C0C0}Valor total} \\ \hline
 &
  \multicolumn{1}{c|}{} &
  \multicolumn{1}{c|}{} &
  \multicolumn{1}{c|}{} \\ \hline
 &
  \multicolumn{1}{c|}{} &
  \multicolumn{1}{c|}{} &
  \multicolumn{1}{c|}{} \\ \hline
\multicolumn{1}{|l|}{} &
   &
   &
   \\ \hline
\multicolumn{1}{|l|}{} &
   &
   &
   \\ \hline
\multicolumn{3}{|c|}{SUBTOTAL} &
  \multicolumn{1}{c|}{} \\ \hline
\rowcolor[HTML]{C0C0C0} 
\multicolumn{4}{|c|}{\cellcolor[HTML]{C0C0C0}COSTOS INDIRECTOS} \\ \hline
\rowcolor[HTML]{C0C0C0} 
Descripción &
  \multicolumn{1}{c|}{\cellcolor[HTML]{C0C0C0}Cantidad} &
  \multicolumn{1}{c|}{\cellcolor[HTML]{C0C0C0}Valor unitario} &
  \multicolumn{1}{c|}{\cellcolor[HTML]{C0C0C0}Valor total} \\ \hline
\multicolumn{1}{|l|}{} &
   &
   &
   \\ \hline
\multicolumn{1}{|l|}{} &
   &
   &
   \\ \hline
\multicolumn{1}{|l|}{} &
   &
   &
   \\ \hline
\multicolumn{3}{|c|}{SUBTOTAL} &
  \multicolumn{1}{c|}{} \\ \hline
\rowcolor[HTML]{C0C0C0}
\multicolumn{3}{|c|}{TOTAL} &
   \\ \hline
\end{tabularx}%
\end{table}


\section{11. Matriz de asignación de responsabilidades}
\label{sec:responsabilidades}
\begin{consigna}{red}
Establecer la matriz de asignación de responsabilidades y el manejo de la autoridad completando la siguiente tabla:

\begin{table}[htpb]
\centering
\resizebox{\textwidth}{!}{%
\begin{tabular}{|c|c|c|c|c|c|}
\hline
\rowcolor[HTML]{C0C0C0} 
\cellcolor[HTML]{C0C0C0} &
  \cellcolor[HTML]{C0C0C0} &
  \multicolumn{4}{c|}{\cellcolor[HTML]{C0C0C0}Listar todos los nombres y roles del proyecto} \\ \cline{3-6} 
\rowcolor[HTML]{C0C0C0} 
\cellcolor[HTML]{C0C0C0} &
  \cellcolor[HTML]{C0C0C0} &
  Responsable &
  Orientador &
  Equipo &
  Cliente \\ \cline{3-6} 
\rowcolor[HTML]{C0C0C0} 
\multirow{-3}{*}{\cellcolor[HTML]{C0C0C0}\begin{tabular}[c]{@{}c@{}}Código\\ WBS\end{tabular}} &
  \multirow{-3}{*}{\cellcolor[HTML]{C0C0C0}Nombre de la tarea} &
  \authorname &
  \supname &
  Nombre de alguien &
  \clientename \\ \hline
 &  &  &  &  &  \\ \hline
 &  &  &  &  &  \\ \hline
 &  &  &  &  &  \\ \hline
\end{tabular}%
}
\end{table}

{\footnotesize
Referencias:
\begin{itemize}
	\item P = Responsabilidad Primaria
	\item S = Responsabilidad Secundaria
	\item A = Aprobación
	\item I = Informado
	\item C = Consultado
\end{itemize}
} %footnotesize

Una de las columnas debe ser para el Director, ya que se supone que participará en el proyecto.
A su vez se debe cuidar que no queden muchas tareas seguidas sin ``A'' o ``I''.

Importante: es redundante poner ``I/A'' o ``I/C'', porque para aprobarlo o responder consultas primero la persona debe ser informada.

\end{consigna}

\section{12. Gestión de riesgos}
\label{sec:riesgos}

\begin{consigna}{red}
a) Identificación de los riesgos (al menos cinco) y estimación de sus consecuencias:
 
Riesgo 1: detallar el riesgo (riesgo es algo que si ocurre altera los planes previstos)
\begin{itemize}
\item Severidad (S): mientras más severo, más alto es el número (usar números del 1 al 10).\\
Justificar el motivo por el cual se asigna determinado número de severidad (S).
\item Probabilidad de ocurrencia (O): mientras más probable, más alto es el número (usar del 1 al 10).\\
Justificar el motivo por el cual se asigna determinado número de (O). 
\end{itemize}   

Riesgo 2:
\begin{itemize}
\item Severidad (S): 
\item Ocurrencia (O):
\end{itemize}

Riesgo 3:
\begin{itemize}
\item Severidad (S): 
\item Ocurrencia (O):
\end{itemize}


b) Tabla de gestión de riesgos:      (El RPN se calcula como RPN=SxO)

\begin{table}[htpb]
\centering
\begin{tabularx}{\linewidth}{@{}|X|c|c|c|c|c|c|@{}}
\hline
\rowcolor[HTML]{C0C0C0} 
Riesgo & S & O & RPN & S* & O* & RPN* \\ \hline
       &   &   &     &    &    &      \\ \hline
       &   &   &     &    &    &      \\ \hline
       &   &   &     &    &    &      \\ \hline
       &   &   &     &    &    &      \\ \hline
       &   &   &     &    &    &      \\ \hline
\end{tabularx}%
\end{table}

Criterio adoptado: 
Se tomarán medidas de mitigación en los riesgos cuyos números de RPN sean mayores a...

Nota: los valores marcados con (*) en la tabla corresponden luego de haber aplicado la mitigación.

c) Plan de mitigación de los riesgos que originalmente excedían el RPN máximo establecido:
 
Riesgo 1: plan de mitigación (si por el RPN fuera necesario elaborar un plan de mitigación).
  Nueva asignación de S y O, con su respectiva justificación:
  - Severidad (S): mientras más severo, más alto es el número (usar números del 1 al 10).
          Justificar el motivo por el cual se asigna determinado número de severidad (S).
  - Probabilidad de ocurrencia (O): mientras más probable, más alto es el número (usar del 1 al 10).
          Justificar el motivo por el cual se asigna determinado número de (O).

Riesgo 2: plan de mitigación (si por el RPN fuera necesario elaborar un plan de mitigación).
 
Riesgo 3: plan de mitigación (si por el RPN fuera necesario elaborar un plan de mitigación).

\end{consigna}


\section{13. Gestión de la calidad}
\label{sec:calidad}

\begin{consigna}{red}
Para cada uno de los requerimientos del proyecto indique:
\begin{itemize} 
\item Req \#1: copiar acá el requerimiento.

Verificación y validación:

\begin{itemize}
\item Verificación para confirmar si se cumplió con lo requerido antes de mostrar el sistema al cliente. Detallar 
\item Validación con el cliente para confirmar que está de acuerdo en que se cumplió con lo requerido. Detallar  
\end{itemize}

\end{itemize}

Tener en cuenta que en este contexto se pueden mencionar simulaciones, cálculos, revisión de hojas de datos, consulta con expertos, mediciones, etc.

\end{consigna}

\section{14. Comunicación del proyecto}
\label{sec:comunicaciones}

El plan de comunicación del proyecto es el siguiente:

\begin{table}[htpb]
\centering
\begin{tabularx}{\linewidth}{@{}|X|C{2.4cm}|C{3cm}|C{1.8cm}|C{2cm}|C{2.1cm}|@{}}
\hline
\rowcolor[HTML]{C0C0C0} 
\multicolumn{6}{|c|}{\cellcolor[HTML]{C0C0C0}PLAN DE COMUNICACIÓN DEL PROYECTO}           \\ \hline
\rowcolor[HTML]{C0C0C0} 
¿Qué comunicar? & Audiencia & Propósito & Frecuencia & Método de comunicac. & Responsable \\ \hline
                &           &           &            &                      &             \\ \hline
                &           &           &            &                      &             \\ \hline
                &           &           &            &                      &             \\ \hline
                &           &           &            &                      &             \\ \hline
                &           &           &            &                      &             \\ \hline
\end{tabularx}
\end{table}

\section{15. Gestión de compras}
\label{sec:compras}

\begin{consigna}{red}
En caso de tener que comprar elementos o contratar servicios:
a) Explique con qué criterios elegiría a un proveedor.
b) Redacte el Statement of Work correspondiente.
\end{consigna}

\section{16. Seguimiento y control}
\label{sec:seguimiento}

\begin{consigna}{red}
Para cada tarea del proyecto establecer la frecuencia y los indicadores con los se seguirá su avance y quién será el responsable de hacer dicho seguimiento y a quién debe comunicarse la situación (en concordancia con el Plan de Comunicación del proyecto).

El indicador de avance tiene que ser algo medible, mejor incluso si se puede medir en \% de avance. Por ejemplo,se pueden indicar en esta columna cosas como ``cantidad de conexiones ruteadeas'' o ``cantidad de funciones implementadas'', pero no algo genérico y ambiguo como ``\%'', porque el lector no sabe porcentaje de qué cosa.

\end{consigna}

\begin{longtable}{|m{1cm}|m{3.5cm}|m{2.2cm}|m{2cm}|m{3cm}|m{1.5cm}|}
\hline
\rowcolor[HTML]{C0C0C0} 
\multicolumn{6}{|c|}{\cellcolor[HTML]{C0C0C0}SEGUIMIENTO DE AVANCE}                                                                       \\ \hline
\rowcolor[HTML]{C0C0C0} 
Tarea del WBS 			& Indicador de avance & Frecuencia de reporte & Resp. de seguimiento & Persona a ser informada & Método de comunic. \\ \hline
\endfirsthead

\hline
\rowcolor[HTML]{C0C0C0} 
\multicolumn{6}{c}{\cellcolor[HTML]{C0C0C0}SEGUIMIENTO DE AVANCE}                                                                       \\ \hline
\rowcolor[HTML]{C0C0C0} 
Tarea del WBS 			& Indicador de avance & Frecuencia de reporte & Resp. de seguimiento & Persona a ser informada & Método de comunic. \\ \hline
\endhead

\multicolumn{6}{c}{Continúa}
\endfoot

\endlastfoot

1.1	& Fecha de inicio  & Única vez al comienzo & \authorname & \clientename, \supname & email \\ \hline
2.1	& Avance de las subtareas  & Mensual mientras dure la tarea & \authorname & \clientename, \supname & email \\ \hline

\end{longtable}

\begin{table}[!htpb]
\centering
%\begin{tabularx}{\linewidth}{@{}|X|X|X|X|X|X|@{}}
\begin{tabularx}{\linewidth}{@{}|X|C{2.5cm}|C{3cm}|C{2cm}|C{2cm}|C{2.5cm}|@{}}
\hline
\rowcolor[HTML]{C0C0C0} 
\multicolumn{6}{|c|}{\cellcolor[HTML]{C0C0C0}SEGUIMIENTO DE AVANCE}                                                                       \\ \hline
\rowcolor[HTML]{C0C0C0} 
Tarea del WBS & Indicador de avance & Frecuencia de reporte & Resp. de seguimiento & Persona a ser informada & Método de comunic. \\ \hline
 &  &  &  &  &  \\ \hline
 &  &  &  &  &  \\ \hline
 &  &  &  &  &  \\ \hline
 &  &  &  &  &  \\ \hline
 &  &  &  &  &  \\ \hline
\end{tabularx}%
%}
\end{table}

\section{17. Procesos de cierre}    
\label{sec:cierre}

\begin{consigna}{red}
Establecer las pautas de trabajo para realizar una reunión final de evaluación del proyecto, tal que contemple las siguientes actividades:

\begin{itemize}
\item Pautas de trabajo que se seguirán para analizar si se respetó el Plan de Proyecto original:
 - Indicar quién se ocupará de hacer esto y cuál será el procedimiento a aplicar. 
\item Identificación de las técnicas y procedimientos útiles e inútiles que se utilizaron, y los problemas que surgieron y cómo se solucionaron:
 - Indicar quién se ocupará de hacer esto y cuál será el procedimiento para dejar registro.
\item Indicar quién organizará el acto de agradecimiento a todos los interesados, y en especial al equipo de trabajo y colaboradores:
  - Indicar esto y quién financiará los gastos correspondientes.
\end{itemize}

\end{consigna}


\end{document}
